
% Zachary Teper
% MASc Thesis Defence Presentation

\documentclass{beamer}
\usetheme{Madrid}
\usecolortheme{default}

\setbeamertemplate{caption}[numbered]
\setbeamertemplate{bibliography item}{\insertbiblabel}

% Import packages
\usepackage{upgreek}
\usepackage{amsmath}
\usepackage{physics}
\usepackage[english]{babel}
\usepackage{dirtytalk}
\usepackage{graphicx}
\usepackage{textcomp}
\usepackage{array}
%\usepackage{biblatex}

% Define macros
\newcommand{\degrees}{$\,^{\circ}$}
\newcommand{\degreesC}{$\,^{\circ}\textrm{C}$}
\newcommand{\registered}{\textsuperscript{\textregistered}\,}
\newcommand{\microamps}{$\upmu$A}
\newcommand{\microns}{$\upmu$m}
\newcommand{\first}{1\textsuperscript{st}}
\newcommand{\second}{2\textsuperscript{nd}}

% Define fixed width column
\newcolumntype{P}[1]{>{\centering\arraybackslash}p{#1}}

% Import images
\graphicspath{
{../ut-thesis/images/introduction}
{../ut-thesis/images/background}
{../ut-thesis/images/autobias_system}
{../ut-thesis/images/dco}
{../ut-thesis/images/lc_resonator}
{../ut-thesis/images/qc_general}
{../ut-thesis/plots/autobias_system/simulation}
{../ut-thesis/plots/autobias_system/measurement}
{../ut-thesis/plots/lc_resonator/simulation}
{../ut-thesis/plots/lc_resonator/measurement}
{../ut-thesis/plots/zt_dco/simulation}
{../ut-thesis/plots/zt_dco/measurement}
}

\AtBeginSection[]{
\begin{frame}
\vfill
\centering
\begin{beamercolorbox}[sep=8pt,center,shadow=true,rounded=true]{title}
\usebeamerfont{title}\insertsectionhead\par
\end{beamercolorbox}
\vfill
\end{frame}
}

% Content
\title[MASc Thesis Defence]
{Millimetre-Wave Integrated Circuits for Qubit Spin Control \& Readout in Silicon Quantum Processors}

\subtitle{MASc Thesis Defence}

\author{Zachary Teper}

\institute[UofT] {
Edward S. Rogers Department of Electrical \& Computer Engineering \\
University of Toronto \\
Supervisor: Prof. Sorin Voinigescu
}

\date[September 19\textsuperscript{th}, 2024]{September 19\textsuperscript{th}, 2024}

% UofT logo
\titlegraphic{\includegraphics[width=2cm]
{logo_uoft}}

\definecolor{uoftblue}{RGB}{6,41,88}
\setbeamercolor{titlelike}{bg=uoftblue}
\setbeamerfont{title}{series=\bfseries}

\begin{document}

\frame{\titlepage}

\begin{frame}
\frametitle{Agenda}
\tableofcontents
\end{frame}

\section{Introduction \& Background}
% 5 slides

\begin{frame}
\frametitle{Quantum Computing Using Semiconductor Spin Qubits}
\begin{columns}[t]

\begin{column}{0.5\textwidth}
\begin{figure}[htpb]
\centering
\includegraphics[width=0.9\textwidth]{bloch_sphere.png}
\caption{Bloch Sphere \cite{oliver_engineers_guide_to_qubits}}
\label{fig:bloch_sphere}
\end{figure}
\mbox{[Krantz et. al, 2019]}
\end{column}

\begin{column}{0.5\textwidth}
\begin{figure}[htpb]
\centering
\includegraphics[width=0.9\textwidth]{two_qubit_qpu_micrograph.png}
\caption{Two-qubit false-colour QPU micrograph \cite{petta_two_qubit_qpu}}
\label{fig:two_qubit_qpu_micrograph}
\end{figure}
\mbox{[Mills et. al, 2022]}
\end{column}

\end{columns}
\end{frame}

\begin{frame}
\frametitle{Objectives}
Objectives:
\begin{enumerate}
\item Explore several FDSOI CMOS circuit designs for qubit spin manipulation and readout
\item Show measurement and simulation results to justify their worth
\item  Enable the integration of very large numbers of qubits on a QPU together with their accompanying
electronics, emphasizing scalability
\end{enumerate}
\end{frame}

\begin{frame}
\frametitle{Spin-Quantum Algorithms}
\begin{figure}[htpb]
\centering
\includegraphics[width=0.7\textwidth]{two_qubit_algorithm.png}
\caption{Example of a quantum algorithm with two qubits \cite{intel_ims_slides}}
\label{fig:two_qubit_algorithm}
\end{figure}
\mbox{[Pellerano \& Subramanian, 2024]}
\end{frame}

\begin{frame}
\frametitle{Semiconductor Spin Quantum Computing}
\begin{columns}[t]

\begin{column}{0.5\textwidth}
\begin{figure}[htpb]
\centering
\includegraphics[width=0.9\textwidth]{two_qubit_spin_control.png}
\caption{Two qubit spin control, drawn on a Bloch Sphere \cite{intel_ims_slides}}
\label{fig:two_qubit_spin_control}
\end{figure}
\mbox{[Pellerano \& Subramanian, 2024]}
\end{column}

\begin{column}{0.5\textwidth}
\begin{figure}[htpb]
\centering
\includegraphics[width=0.9\textwidth]{qpc_reflectometry.png}
\caption{One-port reflectometry experiment on a quantum point contact coupled to a double quantum dot \cite{gossard_qpc_charge_sensing}}
\label{fig:qpc_reflectometry}
\end{figure}
\mbox{[Reilly et. al, 2007]}
\end{column}

\end{columns}
\end{frame}

\begin{frame}
\frametitle{Quantum Computer System Architecture}
\begin{figure}[htpb]
\centering
\includegraphics[width=0.7\textwidth]{cryo_cmos_block_diagram.png}
\caption{Block diagram of current and future configurations for scalable quantum computers with cryogenic integrated circuits \cite{intel_ims_slides}}
\label{fig:cryo_cmos_block_diagram}
\end{figure}
\mbox{[Pellerano \& Subramanian, 2024]}
\end{frame}

\begin{frame}
\frametitle{Fully-Depleted Silicon-On-Insulator Technology}
\begin{figure}[htpb]
\centering
\includegraphics[width=0.7\textwidth]{22fdx_cross_section.png}
\caption{Cross section of NMOS and PMOS transistors in 22FDX\registered \cite{sv_22fdx_unique_features}}
\label{fig:22fdx_cross_section}
\end{figure}
\mbox{[Bonen et. al, 2022]}
\end{frame}

\begin{frame}
\frametitle{Fully-Depleted Silicon-On-Insulator Technology}
\begin{figure}[htpb]
\centering
\includegraphics[width=0.9\textwidth]{vt_vs_vbg.png}
\caption{Measured threshold voltage versus backgate voltage for a 40 x 18 nm x 590 nm transistor \cite{sv_22fdx_unique_features}}
\label{fig:vt_vs_vbg}
\end{figure}
\mbox{[Bonen et. al, 2022]}
\end{frame}

\section{Temperature Insensitive DC Auto-Biasing in FDSOI CMOS}
% 5 slides

\begin{frame}
\frametitle{CMOS Inverters: The Basic Circuit Building Block}
\begin{figure}[htpb]
\centering
\includegraphics[width=0.7\textwidth]{sb_qubit_readout_tia.png}
\caption{Schematic of a qubit readout TIA; an example of an analog amplifier which relies on CMOS inverter-based circuits and must work reliably at cryogenic temperatures \cite{shai_masc_thesis}}
\label{fig:sb_qubit_readout_tia}
\end{figure}
\mbox{[Bonen, 2020]}
\end{frame}

\begin{frame}
\frametitle{CMOS Inverter Biasing Requirements}
The devices sizes and bias voltages must be designed such that \cite{lucy_thesis}:
\begin{enumerate}
\item The input and output voltages (i.e. the devices' V\textsubscript{GS} and V\textsubscript{DS}, respectively), are equal to V\textsubscript{dd}/2
\item Both the NMOS and PMOS transistors are in the saturation regime
\item The two devices have the same \say{drive strength}. That is, they have the same on-current ($I_{\textrm{ON}}$) such that for a given capacitive load, the PMOS and NMOS transistors can charge and discharge the load at the same speed
\item The devices are both biased at the peak $\textrm{f}_{max}$\footnote[1]{$\textrm{f}_{T}$ or $\textrm{NF}_{min}$ may also be used} current density
\item The devices have the same threshold voltage
\end{enumerate}
\end{frame}

\begin{frame}
\frametitle{CMOS Current Matching using Backgates}
\begin{figure}[htpb]
\centering
\includegraphics[scale=0.5]{lucy_backgate_circuit.png}
\caption{Backgate setting replica circuit \cite{lucy_thesis}}
\label{fig:lucy_backgate_circuit}
\end{figure}
\mbox{[Wu, 2022]}
\end{frame}

\begin{frame}
\frametitle{Feedback Circuit for Setting Backgates}
\begin{figure}[htpb]
\centering
\includegraphics[width=0.6\textwidth]{autobias_system_schematic.png}
\caption{Auto-biasing circuit schematic.  $V_{\textrm{ddm}}$ is the replica mirror supply, nominally equal to 0.8 V. $V_{\textrm{ddc}}$ is the target circuit supply, nominally equal to 0.8 V.}
\label{fig:autobias_system_schematic}
\end{figure}
\end{frame}

\begin{frame}
\frametitle{Rail-to-Rail Op Amp Design}
\begin{figure}[htpb]
\centering
\includegraphics[width=0.8\textwidth]{opamp_schematic.png}
\caption{CMOS rail-to-rail op amp schematic. All devices have 300 nm gate lengths. All NMOS devices have 1.8 V backgates, and all PMOS devices have 0 V backgates. All transistors are of the EGSLVT (thick-oxide, maximum 1.8 V, super-low $|V_{\textrm{t}}|$) flavour.}
\label{fig:opamp_schematic}
\end{figure}
\end{frame}

\begin{frame}
\frametitle{Rail-to-Rail Op Amp Design}
\begin{figure}[htpb]
\centering
\includegraphics[width=0.8\textwidth]{opamp_operating_point.png}
\caption{CMOS rail-to-rail op amp operating points. All devices have 300 nm gate lengths. All NMOS devices have 1.8 V backgates, and all PMOS devices have 0 V backgates. All transistors are of the EGSLVT (thick-oxide, maximum 1.8 V, super-low $|V_{\textrm{t}}|$) flavour.}
\label{fig:opamp_operating_point}
\end{figure}
\end{frame}

\begin{frame}
\frametitle{Laboratory Measurements}
\begin{columns}[t]

\begin{column}{0.5\textwidth}
\begin{figure}[htpb]
\centering
\includegraphics[width=0.9\textwidth]{jds_vs_independent_temp_0p8V.png}
\caption[Measured target circuit current density versus temperature]{Measured target circuit current density versus temperature, with 0.8 V supply voltage.}
\label{fig:jds_vs_independent_temp_0p8V}
\end{figure}
\end{column}

\begin{column}{0.5\textwidth}
\begin{figure}[htpb]
\centering
\includegraphics[width=0.9\textwidth]{voltage_vs_independent_temp_0p8V.png}
\caption[Measured target circuit output voltage versus temperature]{Measured target circuit output voltage versus temperature, with 0.8 V supply voltage.}
\label{fig:voltage_vs_independent_temp_0p8V}
\end{figure}
\end{column}

\end{columns}
\end{frame}

\begin{frame}
\frametitle{Laboratory Measurements}
\begin{columns}[t]

\begin{column}{0.5\textwidth}
\begin{figure}[htpb]
\centering
\includegraphics[width=0.9\textwidth]{jds_vs_independent_temp_0p8V_sim_vs_measurement.png}
\caption{Comparison of simulated and measured current density of the target circuit, for $J_{mirror}$ = 250 $\upmu$A/$\upmu$m and 0.8 V supply voltage}
\label{fig:jds_vs_independent_temp_0p8V_sim_vs_measurement}
\end{figure}
\end{column}

\begin{column}{0.5\textwidth}
\begin{figure}[htpb]
\centering
\includegraphics[width=0.9\textwidth]{vout_vs_independent_temp_0p8V_sim_vs_measurement.png}
\caption{Comparison of simulated and measured output voltage of the target circuit, for $J_{mirror}$ = 250 $\upmu$A/$\upmu$m and 0.8 V supply voltage}
\label{fig:vout_vs_independent_temp_0p8V_sim_vs_measurement}
\end{figure}
\end{column}

\end{columns}
\end{frame}

\section{LC Resonant Circuits for Qubit Spin Readout}
% 5 slides

\begin{frame}
\frametitle{Reflectometry Readout Techniques \& Requirements}
\begin{itemize}
\item After a spin-quantum computation has finished, the results must be read by collapsing the qubits' wave functions into either spin-up or spin-down states
\item Exploiting physical phenomena such as quantum tunnelling and Zeeman splitting allows these quantum mechanical results to manifest as electrical parameters such as resistance and capacitance, which can be measured using classical RF techniques
\item Dispersive readout, which measures the change in capacitance of a charge-sensing SET, relies on the Pauli spin blockade effect to convert a qubit's spin into the presence/absence of charges in the SET \cite{vigneau_review}
\end{itemize}
\end{frame}

\begin{frame}
\frametitle{Reflectometry Readout Techniques \& Requirements}

\begin{equation}
\label{eqn:set_capacitance}
C_s = C_{geometry} + C_{quantum} + C_{tunnelling}
\end{equation}

\begin{equation}
\label{eqn:quantum_capacitance}
C_{quantum} = q^2 \frac{\partial N}{\partial \mu} = q^2 \lim_{T\to 0}  \rho(E_F)
\end{equation}

\begin{equation}
\label{eqn:tunnelling_capacitance1}
C_{tunnelling} = \frac{(q\alpha)^2}{k_B T} \frac{R_K}{R_T} \frac{\epsilon_0}{h\gamma} \frac{\gamma^2}{\gamma^2 + \omega^2}\textrm{arsinh}(\frac{\epsilon_0}{k_B T})
\end{equation}

\begin{equation}
\label{eqn:tunnelling_capacitance2}
\gamma = \frac{R_K}{R_T} \frac{\epsilon_0}{h} \textrm{coth}(\frac{\epsilon_0}{2k_B T})
\end{equation}

\mbox{\cite{vigneau_review}}
\end{frame}

\begin{frame}
\frametitle{Reflectometry System Architecture}
\begin{figure}[htpb]
\centering
\includegraphics[width=0.8\textwidth]{sv_iqubits_reflectometry.png}
\caption{Reflectometry architecture for a 1-dimensional qubit array, with a dedicated inductor for each qubit \cite{sv_iqubits_report}}
\label{fig:sv_iqubits_reflectometry}
\end{figure}
\mbox{[Voinigescu, 2024]}
\end{frame}

\begin{frame}
\frametitle{Reflectometry Test Circuit}
\begin{figure}[htpb]
\centering
\includegraphics[height=0.6\textheight]{lc_resonator_schematic.png}
\caption{LC resonator breakout schematic. The red boxes indicate the transmission lines and test pads that are de-embedded. The green box indicates the DUT circuit for which S-parameters are extracted after measurement with a VNA.}
\label{fig:lc_resonator_schematic}
\end{figure}
\end{frame}

\begin{frame}
\frametitle{12-bit Minimum Sized Unit Varactor Array}
\begin{columns}[t]
\begin{column}{0.5\textwidth}
\begin{figure}[htpb]
\centering
\includegraphics[width=0.9\textwidth]{varactor_12bit_layout.png}
\caption{Layout of the entire 12-bit varactor}
\label{fig:varactor_12bit_layout}
\end{figure}
\end{column}

\begin{column}{0.5\textwidth}
\begin{figure}[htpb]
\centering
\includegraphics[width=0.9\textwidth]{varactor_4bit_layout.png}
\caption{Layout of the lowest 4 bits of the 12-bit varactor}
\label{fig:varactor_4bit_layout}
\end{figure}
\end{column}
\end{columns}
\end{frame}

\begin{frame}
\frametitle{12-bit Minimum Sized Unit Varactor Array}
\begin{figure}[htpb]
\centering
\includegraphics[width=0.5\textwidth]{zt_unit_varactor.png}
\caption{Simulation (without extracted parasitics) of the 12-bit varactor when adjusting the control voltage of the first 2 LSBs. The overall change in capacitance caused by the \first \, LSB is 18.4 aF. The overall change in capacitance caused by the \second \, LSB is 37.5 aF. In this simulation, the backgate voltage is zero, the top gate voltage is 0.4 V, and the 10 MSBs are set to 0.8 V.}
\label{fig:unit_varactor}
\end{figure}
\end{frame}

\begin{frame}
\frametitle{Phase Sensitivity}
\begin{equation}
\label{eqn:phase_definition}
\phi = phase\{-Y_{12}\}
\end{equation}

\begin{equation}
\label{eqn:phase_sensitivity_quantum}
Phase\:Sensitivity\:(f) = \frac{\phi_{\ket{\uparrow}}(f) - \phi_{\ket{\downarrow}}(f)}{C_{\ket{\uparrow}} - C_{\ket{\downarrow}}} = \frac{\Delta \phi}{\Delta C}(f)
\end{equation}

\begin{equation}
\label{eqn:phase_sensitivity_varactor}
Phase\:Sensitivity\:(f) = \frac{\phi_{+0.8\:V}(f) - \phi_{-0.4\:V}(f)}{C_{+0.8\:V} - C_{-0.4\:V}} = \frac{\Delta \phi}{\Delta C}(f)
\end{equation}

\begin{equation}
\label{eqn:parallel_rlc_ps_result}
Phase\:Sensitivity = \boxed{PS(\omega_r) = -\omega_r R = -Q/C}
\end{equation}
\end{frame}

\begin{frame}
\frametitle{Measured Resonance Frequency}
\begin{figure}[htpb]
\centering
\includegraphics[width=0.5\textwidth]{zt_measured_tuning_range.png}
\caption{Measured and simulated tuning range, with respect to the MSB control voltage. The measured tuning range is 47.45 GHz to 63.40 GHz (28.8 \%).}
\label{fig:measured_tuning_range}
\end{figure}
\end{frame}

\begin{frame}
\frametitle{Measured Phase Sensitivity}
\begin{columns}[t]
\begin{column}{0.5\textwidth}
\begin{figure}[htpb]
\centering
\includegraphics[width=0.9\textwidth]{zt_phase_b1_measured.png}
\caption{Measured LC resonator phase with respect to the \first \, LSB control voltage.}
\label{fig:phase_b1_measured}
\end{figure}
\end{column}

\begin{column}{0.5\textwidth}
\begin{figure}[htpb]
\centering
\includegraphics[width=0.9\textwidth]{zt_phase_sensitivity_b1_measured.png}
\caption{Measured PS with respect to frequency and the \first \, LSB control voltage. The PS is calculated by dividing the difference in phase between the highest and lowest voltage by 20 aF}
\label{fig:phase_sensitivity_b1_measured}
\end{figure}
\end{column}

\end{columns}
\end{frame}

\begin{frame}
\frametitle{Measured Phase Sensitivity}
\begin{columns}[t]
\begin{column}{0.5\textwidth}
\begin{figure}[htpb]
\centering
\includegraphics[width=0.9\textwidth]{zt_phase_b2_measured.png}
\caption{Measured LC resonator phase with respect to the \second \, LSB control voltage.}
\label{fig:phase_b2_measured}
\end{figure}
\end{column}

\begin{column}{0.5\textwidth}
\begin{figure}[htpb]
\centering
\includegraphics[width=0.9\textwidth]{zt_phase_sensitivity_b2_measured.png}
\caption{Measured PS with respect to frequency and the \second \, LSB control voltage. The PS is calculated by dividing the difference in phase between the highest and lowest voltage by 40 aF}
\label{fig:phase_sensitivity_b2_measured}
\end{figure}
\end{column}

\end{columns}
\end{frame}

\begin{frame}
\frametitle{Comparison to Prior Art}
\begin{table}[htpb]
\centering
\begin{tabular}{|P{3cm}|P{2cm}|P{2cm}|P{2cm}|}
\hline
Specification & \textbf{This Work} & \cite{impedancemetry_leGuevel} & \cite{dispersive_readout_franceschi} \\
\hline
Resonance Frequency [MHz] & \textbf{65500} & 190 & 430 \\
\hline
Peak Phase Sensitivity [\degrees/fF] & \textbf{~45} & 1.7 & ~0.00252 \\
\hline
Temperature [K] & \textbf{293} & 4.2 & 0.35 \\
\hline
Inductor Type & \textbf{Monolithic Integrated} & Active Gm-C &  Surface-Mounted Ceramic Chip \\
\hline
Integration Time [s] & \textbf{430} & 1 & 0.000005 \\
\hline
Peak Quality Factor & \textbf{8} & 257 & ~100 \\
\hline
\end{tabular}
\label{tab:reflectometry_comparison}
\end{table}
\end{frame}

\section{Frequency Generation for Qubit Spin Control}
% 5 slides

\begin{frame}
\frametitle{Frequency Generation for Qubit Spin Manipulation}
\begin{itemize}
\item Spin manipulation of qubits involves applying microwave pulses at frequencies close to the qubits' Larmor frequencies in order to perform logical operations (i.e. quantum gates) on them to execute a quantum algorithm \cite{vandersypen_qubit_spin_control}
\item Fine-grained control of the oscillation frequency is desirable since it enables the frequency source to tune into the resonance frequency of a particular group of qubits, and compensate for variations in the resonance frequency that may result from imperfections in the qubit structures
\end{itemize}
\end{frame}

\begin{frame}
\frametitle{Qubit Spin Control Frequency Source Requirements}
The frequency source must:
\begin{enumerate}
\item Have low phase noise to prevent phase errors \cite{edoardo_cryo_cmos}
\item Be frequency stable in order to avoid frequency drift which causes phase errors in both the driven qubit and qubits tuned to adjacent frequencies
\item Be tunable over a wide range in order to address a large array of qubits, where each qubit has a slightly different (but not necessarily unique) resonance frequency \cite{pellerano_cryo_cmos}
\end{enumerate}
\end{frame}

\begin{frame}
\frametitle{Oscillator Tuning Range}
\begin{equation}
Varactor\:Ratio = VR = \frac{\Delta C}{C_{fixed}}
\end{equation}

\begin{equation}
Tuning\:Range = TR = 2\frac{\omega_{max} - \omega_{min}}{\omega_{max} + \omega_{min}}
\end{equation}

\begin{equation}
\label{eqn:tuning_range_fcn_varactor_ratio}
\boxed{TR = 2\frac{\sqrt{VR+1} - 1}{\sqrt{VR+1} + 1}}
\end{equation}
\end{frame}

\begin{frame}
\frametitle{Design of a CMOS Cross-Coupled DCO}
\begin{figure}[htpb]
\centering
\includegraphics[height=0.8\textheight]{dco_schematic.png}
\label{fig:dco_schematic}
\end{figure}
\end{frame}

\begin{frame}
\frametitle{CMOS Buffers}
\begin{figure}[htpb]
\centering
\includegraphics[width=0.7\textwidth]{cmos_buffer_schematic.png}
\caption{CMOS buffer schematic showing the DC operating point}
\label{fig:cmos_buffer_schematic}
\end{figure}
\end{frame}

\begin{frame}
\frametitle{Deserializer}
\begin{figure}[htpb]
\centering
\includegraphics[width=0.8\textwidth]{new_deserializer_schematic.png}
\caption{New deserializer schematic. This deserializer design was not fabricated nor tested in the laboratory, but simulations indicate that the new design eliminates the metastability problem found in the previous iterations of the deserializer. The \say{12f} CMOS inverters are made of transistors with 12 fingers}
\label{fig:new_deserializer_schematic}
\end{figure}
\end{frame}

\begin{frame}
\frametitle{DCO Non-Linearity}
\begin{figure}[htpb]
\centering
\includegraphics[width=0.5\textwidth]{dco_tuning_range.png}
\caption{Simulated DCO tuning range, for a set of 16 uniformly spaced digital codes ranging from 0x000 ($f_{osc}$ = 61.34 GHz) to 0xFFF ($f_{osc}$ = 69.43 GHz). The simulated oscillation frequency increases by roughly 2 MHz per LSB step. The best fit line is computed by the least squares method}
\label{fig:dco_tuning_range}
\end{figure}
\end{frame}

\begin{frame}
\frametitle{DCO Non-Linearity}
\begin{columns}[t]
\begin{column}{0.5\textwidth}
\begin{figure}[t]
\centering
\includegraphics[width=0.9\textwidth]{dco_inl_dnl.png}
\caption{INL and DNL in terms of frequency. The INL is equal to the difference between the best fit line and the simulated frequency at each code}
\label{fig:dco_inl_dnl}
\end{figure}
\end{column}

\begin{column}{0.5\textwidth}
\begin{figure}[t]
\centering
\includegraphics[width=0.9\textwidth]{dco_inl_dnl_lsb.png}
\caption{INL and DNL in terms of the LSB step}
\label{fig:dco_inl_dnl_lsb}
\end{figure}
\end{column}
\end{columns}
\end{frame}

\begin{frame}
\frametitle{Measured Spectra}
\begin{columns}[t]
\begin{column}{0.5\textwidth}
\begin{figure}[htpb]
\centering
\includegraphics[width=0.9\textwidth]{zt_dco_spectrum_screenshot_61ghz.png}
\caption{DCO output spectrum when the digital code is set to 0x000. The oscillation frequency is 61.68 GHz}
\label{fig:dco_spectrum_screenshot_61ghz}
\end{figure}
\end{column}

\begin{column}{0.5\textwidth}
\begin{figure}[htpb]
\centering
\includegraphics[width=0.9\textwidth]{zt_dco_spectrum_screenshot_69ghz.png}
\caption{DCO output spectrum when the digital code is set to 0xFFF. The oscillation frequency is 68.92 GHz}
\label{fig:dco_spectrum_screenshot_69ghz}
\end{figure}
\end{column}
\end{columns}
\end{frame}

\begin{frame}
\frametitle{Comparison with Previous DCO Designs}
\begin{table}[htpb]
\centering
\tiny
\begin{tabular}{|P{1.5cm}|P{1.5cm}|P{1.5cm}|P{2cm}|P{1cm}|P{1cm}|}
\hline
Specification & This Work (simulated) & \textbf{This Work (measured)} & Radar Transceiver LO from Alex (measured) & [Chicco et. al, 2023] & [Taha, Mirhassani, 2019] \\
\hline
Minimum Frequency [GHz] & 61.34 & \textbf{61.68} & 54.93 & 18.7 & 75.5 \\
\hline
Maximum Frequency [GHz] & 69.43 & \textbf{68.92} & 62.49 & 24.4 & 82.5 \\
\hline
Absolute Tuning Range [GHz] & 8.09 & \textbf{7.24} & 7.56 & 5.7 & 7.0 \\
\hline
Normalized Tuning Range [\%] & 12.4 & \textbf{9.0} & 12.9 & 27.2 & 8.9 \\
\hline
Power Consumption [mW] & 30.6* & \textbf{35.2*} & 9.0** & 1.2** & 14** \\
\hline
Total Number of Control Bits & 12 & \textbf{12} & 12 & 19 & 15 \\
\hline
Frequency LSB Step [MHz] & 1.98 & \textbf{1.77} & 1.85 & 0.15 & 0.35 \\
\hline
Technology & 22 nm FDSOI CMOS & \textbf{22 nm FDSOI CMOS} & 22 nm FDSOI CMOS & 28 nm bulk CMOS & 65 nm bulk CMOS \\
\hline
\end{tabular}
\mbox{*Power consumption of the entire system, including the buffers, biasing circuit and deserializer.}
\mbox{**Power consumption of only the oscillator core.}
\label{tab:dco_comparison}
\end{table}
\end{frame}

\section{Conclusion}

\begin{frame}
\frametitle{Summary}
\begin{itemize}
\item Further research is needed to enable large-scale quantum computing in silicon
\item This work presents:
\begin{itemize}
\item An auto-biasing circuit topology which can be used over a wide temperature range
\item A resonant circuit which can be used for determining the results of a quantum computation
\item A frequency source which can be used for generating qubit spin manipulation signals
\end{itemize}
\item The circuits were validated with room temperature measurements, and are expected to perform even better at temperatures down to 2 K
\item The circuit design techniques used in traditional analog design can be used to enable quantum computers with ever-increasing numbers of qubits
\end{itemize}
\end{frame}

\begin{frame}
\frametitle{Future Work}
Future generations of students can improve upon this work by:
\begin{itemize}
\item Fabricating and testing the new deserializer design to validate it
\item Designing a new op amp with better linearity and bipolar power supplies to enable the operation of the auto-biasing circuit at cryogenic temperatures
\item Exploring circuit topologies for generating qubit spin manipulation pulses, such as passive switch-based mixers and RFDAC circuits
\end{itemize}
\end{frame}

\begin{frame}
\frametitle{High-Temperature Quantum Computing}
\begin{figure}[htpb]
\centering
\includegraphics[width=0.9\textwidth]{this_is_fine_meme}
\label{fig:this_is_fine_meme}
\end{figure}
\mbox{[Drawn by K.C. Green, edited by Diraq employees]}
\end{frame}

\begin{frame}[allowframebreaks]
\frametitle{References}
\bibliography{zt_thesis}
\bibliographystyle{ieeetr}
\end{frame}

\begin{frame}
\frametitle{Questions}
The presenter is now open to questions from the committee
\end{frame}

\end{document}