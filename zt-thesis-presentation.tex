
% Zachary Teper
% MASc Thesis Defence Presentation

\documentclass{beamer}
\usetheme{Madrid}
\usecolortheme{default}

\setbeamertemplate{caption}[numbered]

% Import packages
\usepackage{upgreek}
\usepackage{amsmath}
\usepackage{physics}
\usepackage[english]{babel}
\usepackage{dirtytalk}
\usepackage{graphicx}
\usepackage{textcomp}

% Define macros
\newcommand{\degrees}{$\,^{\circ}$}
\newcommand{\degreesC}{$\,^{\circ}\textrm{C}$}
\newcommand{\registered}{\textsuperscript{\textregistered}\,}
\newcommand{\microamps}{$\upmu$A}
\newcommand{\microns}{$\upmu$m}
\newcommand{\first}{1\textsuperscript{st}}
\newcommand{\second}{2\textsuperscript{nd}}

% Import images
\graphicspath{
{../ut-thesis/images/introduction}
{../ut-thesis/images/background}
{../ut-thesis/images/autobias_system}
{../ut-thesis/images/dco}
{../ut-thesis/images/lc_resonator}
{../ut-thesis/images/qc_general}
{../ut-thesis/plots/autobias_system/simulation}
{../ut-thesis/plots/autobias_system/measurement}
{../ut-thesis/plots/lc_resonator/simulation}
{../ut-thesis/plots/lc_resonator/measurement}
{../ut-thesis/plots/zt_dco/simulation}
{../ut-thesis/plots/zt_dco/measurement}
}

\AtBeginSection[]{
\begin{frame}
\vfill
\centering
\begin{beamercolorbox}[sep=8pt,center,shadow=true,rounded=true]{title}
\usebeamerfont{title}\insertsectionhead\par
\end{beamercolorbox}
\vfill
\end{frame}
}

% Content
\title[MASc Thesis Defence]
{Millimetre-Wave Integrated Circuits for Qubit Spin Control \& Readout in Silicon Quantum Processors}

\subtitle{MASc Thesis Defence}

\author{Zachary Teper}

\institute[UofT] {
Edward S. Rogers Department of Electrical \& Computer Engineering \\
University of Toronto \\
Supervisor: Prof. Sorin Voinigescu
}

\date[September 19\textsuperscript{th}, 2024]{September 19\textsuperscript{th}, 2024}

\logo{\includegraphics[height=0.8cm]{logo_uoft}}

\definecolor{uoftblue}{RGB}{6,41,88}
\setbeamercolor{titlelike}{bg=uoftblue}
\setbeamerfont{title}{series=\bfseries}

\begin{document}

\frame{\titlepage}

\begin{frame}
\frametitle{Agenda}
\tableofcontents
\end{frame}

\section{Introduction \& Background}
% 5 slides

\begin{frame}
\frametitle{Quantum Computing Using Semiconductor Spin Qubits}
\begin{columns}[c]

\begin{column}{0.5\textwidth}
\begin{figure}[htpb]
\centering
\includegraphics[width=0.9\textwidth]{bloch_sphere.png}
\caption{Bloch Sphere}
\label{fig:bloch_sphere}
\end{figure}
\mbox{[Krantz et. al, 2019]}
\end{column}

\begin{column}{0.5\textwidth}
\begin{figure}[htpb]
\centering
\includegraphics[width=0.9\textwidth]{two_qubit_qpu_micrograph.png}
\caption{Two-qubit false-colour QPU micrograph}
\label{fig:two_qubit_qpu_micrograph}
\end{figure}
\mbox{[Mills et. al, 2022]}
\end{column}

\end{columns}
\end{frame}


\begin{frame}
\frametitle{Objective}
\begin{itemize}
\item Explore several FDSOI CMOS circuit designs for qubit spin manipulation and readout
\item Shows measurement and simulation results to justify their worth
\item  Emphasis is on scalability,
to enable the integration of very large numbers of qubits on a QPU, together with their accompanying
electronics.
\end{itemize}
\end{frame}

\begin{frame}
\frametitle{Semiconductor Spin Quantum Computing}
\begin{columns}[c]

\begin{column}{0.5\textwidth}
\begin{figure}[htpb]
\centering
\includegraphics[width=0.9\textwidth]{two_qubit_spin_control.png}
\caption{Two qubit spin control, drawn on a Bloch Sphere}
\label{fig:two_qubit_spin_control}
\end{figure}
\mbox{[Pellerano \& Subramanian, 2024]}
\end{column}

\begin{column}{0.5\textwidth}
\begin{figure}[htpb]
\centering
\includegraphics[width=0.7\textwidth]{qpc_reflectometry.png}
\caption{One-port reflectometry experiment on a quantum point contact coupled to a double quantum dot}
\label{fig:qpc_reflectometry}
\end{figure}
\mbox{[Reilly et. al, 2007]}
\end{column}

\end{columns}
\end{frame}

\begin{frame}
\frametitle{Quantum Computer System Architecture}
\begin{figure}[htpb]
\centering
\includegraphics[width=0.7\textwidth]{cryo_cmos_block_diagram.png}
\caption{Block diagram of current and future configurations for scalable quantum computers with cryogenic integrated circuits}
\label{fig:cryo_cmos_block_diagram}
\end{figure}
\mbox{[Pellerano \& Subramanian, 2024]}
\end{frame}

\begin{frame}
\frametitle{Fully-Depleted Silicon-On-Insulator Technology}
\begin{figure}[htpb]
\centering
\includegraphics[width=0.7\textwidth]{22fdx_cross_section.png}
\caption{Cross section of NMOS and PMOS transistors in 22FDX\registered}
\label{fig:22fdx_cross_section}
\end{figure}
\mbox{[Bonen et. al, 2022]}
\end{frame}

\begin{frame}
\frametitle{Fully-Depleted Silicon-On-Insulator Technology}
\begin{figure}[htpb]
\centering
\includegraphics[width=0.9\textwidth]{vt_vs_vbg.png}
\caption{Measured threshold voltage versus backgate voltage for a 40 x 18 nm x 590 nm transistor}
\label{fig:vt_vs_vbg}
\end{figure}
\mbox{[Bonen et. al, 2022]}
\end{frame}

\section{Temperature Insensitive DC Auto-Biasing in FDSOI CMOS}
% 5 slides

\begin{frame}
\frametitle{CMOS Inverters: The Basic Circuit Building Block}
CMOS inverters
\end{frame}

\begin{frame}
\frametitle{CMOS Current Matching using Backgates}
backgate current matching
\end{frame}

\begin{frame}
\frametitle{Feedback Circuit for Setting Backgates}
feedback circuit
\end{frame}

\begin{frame}
\frametitle{Rail-to-Rail Op Amp Design}
op amp
\end{frame}

\begin{frame}
\frametitle{Laboratory Measurements}
measurements
\end{frame}

\section{LC Resonant Circuits for Qubit Spin Readout}
% 5 slides

\begin{frame}
\frametitle{Reflectometry Readout Techniques \& Requirements}
reflectometry
\end{frame}

\begin{frame}
\frametitle{12-bit Minimum Sized Unit Varactor Array}
varactor
\end{frame}

\begin{frame}
\frametitle{Phase Response}
phase response
\end{frame}

\begin{frame}
\frametitle{Measured Resonance Frequency}
resonance
\end{frame}

\begin{frame}
\frametitle{Measured Phase Sensitivity}
phase sensitivity
\end{frame}

\section{Frequency Generation for Qubit Spin Control}
% 5 slides

\begin{frame}
\frametitle{Frequency Sources in Quantum Computing}
frequency sources
\end{frame}

\begin{frame}
\frametitle{Oscillator Tuning Range}
tuning range
\end{frame}

\begin{frame}
\frametitle{Design of a CMOS Cross-Coupled DCO}
DCO design
\end{frame}

\begin{frame}
\frametitle{Measured Tuning Range \& Spectra}
tuning range
\end{frame}

\begin{frame}
\frametitle{Comparison with Previous DCO Designs}
comparison
\end{frame}

\section{Conclusion}

\begin{frame}
\frametitle{Conclusion}
conclusion, future work
\end{frame}

\begin{frame}
\frametitle{High-Temperature Quantum Computing}
XXX insert this is fine meme
\end{frame}

\begin{frame}
\frametitle{References}
\begin{enumerate}
\item 1 Krantz
\item 2 Mills
\item 3
\end{enumerate}
\end{frame}

\begin{frame}
\frametitle{Questions}
The presenter is now open to questions from the committee
\end{frame}

%\begin{block}{Remark}
%Sample text
%\end{block}
%
%\begin{alertblock}{Important theorem}
%Sample text in red box
%\end{alertblock}
%
%\begin{examples}
%Sample text in green box. The title of the block is ``Examples".
%\end{examples}

\end{document}